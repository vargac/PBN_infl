\documentclass[
	digital, oneside, nosansbold, nocolorbold, nolot, nolof
]{fithesis4}

\usepackage[resetfonts]{cmap}
\usepackage[T1]{fontenc}
\usepackage[english]{babel}
\usepackage[utf8]{inputenc}

\thesissetup{
    date        = \the\year/\the\month/\the\day,
    university  = mu,
    faculty     = fi,
	department  = {Department of Machine Learning and Data Processing},
    type        = mgr,
    author      = Šimon Varga,
    gender      = m,
    advisor     = {prof. RNDr. Luboš Brim, CSc.},
    title       = {%
		Influence maximization in partially specified Boolean networks},
    keywords    = {keyword1, keyword2, ...},
    abstract    = {%
      This is the abstract of my thesis, which can

      span multiple paragraphs.
    },
    thanks      = {%
      These are the acknowledgements for my thesis, which can

      span multiple paragraphs.
    },
    bib         = bib.bib,
    facultyLogo = fithesis-fi,
}

\usepackage{makeidx}      %% The `makeidx` package contains
\makeindex                %% helper commands for index typesetting.
%% These additional packages are used within the document:
\usepackage{amsmath}  %% Mathematics
\usepackage{amsthm}
\usepackage{amssymb}
\usepackage{amsfonts}
\usepackage{IEEEtrantools}
\usepackage{tikz}
\usetikzlibrary{shapes, positioning}
\usepackage{url}      %% Hyperlinks
\usepackage{listings} %% Source code highlighting
\lstset{
  basicstyle      = \ttfamily,
  identifierstyle = \color{black},
  keywordstyle    = \color{blue},
  keywordstyle    = {[2]\color{cyan}},
  keywordstyle    = {[3]\color{olive}},
  stringstyle     = \color{teal},
  commentstyle    = \itshape\color{magenta},
  breaklines      = true,
}
\usepackage{floatrow} %% Putting captions above tables
\floatsetup[table]{capposition=top}
\usepackage[babel]{csquotes} %% Context-sensitive quotation marks



\begin{document}

\chapter{Introduction}

\paragraph{Systems biology}

A common trait among all humans is a desire to know, to understand. Aristotle
(384--322~BC) described this desire in his work Metaphysics~\cite{aristotle}:

\blockquote[Aristotle]{All men by nature desire to know. An indication of this
    is the delight we take in our senses; for even apart from their usefulness
    they are loved for themselves; and above all others the sense of sight. For
    not only with a view to action, but even when we are not going to do
    anything, we prefer sight to almost everything else. The reason is that
    this, most of all the senses, makes us know and brings to light many
    differences between things.}

A human ambition to see, to explore the world around and inside us has emerged
into many sciences -- one of them being biology, a scientific study of life.
There is a visible trend of \emph{reductionism} interweaving the history of
biology. In the 17th century, René Descartes (1596--1650) stated that complex
systems could be understood by looking into the smaller parts inside the system
and then reassembly them into the whole. Although biology has not yet existed
back then as a scientific discipline of today and the main areas affected by
Descartes' idea of reductionism were mathematics and physics, he also published
his notion of non-human animals being complex automatons designed by a God. He
claimed that such an automaton can be described in a reductionist and
mechanistic way.~\cite{systems_bio_hist, de_homine}

A change in this concept arose in the early part of the 20th century. An
influential work by Norbert Wiener (1894--1964) asked for more system-level
understanding~\cite{cybernetics}.  Many biologists expressed objections against
reductionist attitudes~\cite{woodger_biological, weiss_problem, ludwig_open}. A
new language was used, with dominant terms like complexity, organization,
uniqueness, emergence, unpredictability, and interconnectedness. Reductionistic
and mechanistic biology lacks a view of the vital orchestration of components.
Against reductionism, another paradigm started to be articulated more --
\emph{holism}, coined by Jan Smuts (1870--1950), naturalist, philosopher, and
twice Prime Minister of South Africa~\cite{smuts_holism}.

Although these two theories -- reductionism and holism -- are often put in
contrast to each other, no opposition but reconciliation is needed. It is
important to explore the smaller subsystems, tiny components, and how they
function on their own (reductionism). However, it is not less important to look
at the system from above to explore possible emergent features resulting from
the interplay between components (holism).~\cite{systems_bio_hist}

A nice example is an aggregate motion of a flock of birds.
Reynolds~\cite{reynolds_flock} published a computer simulation where each
agent-bird behaves accordingly to a relatively simple set of rules. A
reductionist scientist could explore these rules and understand the bird's sole
behavior but could not find any evidence among the rules of the amazing
phenomenon because it emerges from the interactions between the components
(birds). A set of simple rules orchestrating the components emerged in the
complex behavior of the whole system -- this is something central to a new
approach in biology rising from holism: \emph{Systems biology}.

To study even a small entity -- a cell -- from the system-wide perspective, one
has to combine interactions of many components on different levels: genome,
transcriptome, proteome, and metabolome space. Processing such a huge amount of
information requires computational methods and software infrastructure,
critical components for systems biology research.~\cite{kitano_overview,
systems_bio_methods}

\paragraph{Boolean networks}

Systems biology models of living systems' dynamics may be categorized as
qualitative or quantitative. An example of a qualitative model is a set of ODE
(ordinary differential equations) precisely describing the kinetics of
processes in the system. Experimental inference of all the kinetic parameters
is usually a complex task. At this place, systems biology benefits from
qualitative models. This work targets one concrete: \emph{Boolean networks}
(BN). BN is a simple model introduced by Stuart Kauffman in 1969 for genetic
regulatory networks (GRN), consisting of simple boolean variables for
components of the GRN (proteins, genes, and metabolites) and discrete relations
between them.  Each variable represents the component being active or not. If
the component is a gene, then the variable represents whether the gene is
transcribed or not. In the case of metabolites or proteins, it may represent
whether the amount of metabolite or protein is sufficient for some reaction.
Boolean functions define the relations between the
variables.~\cite{concepts_bn} Even after such a modeling reduction from the
real-world living system, BNs still may illustrate various processes such as
heart development~\cite{heart_development} and aging of organisms~\cite{aging}.

BNs play an important role in systems biology, but it is often hard to infer
the boolean functions controlling the system. A knowledge that some protein
\(a\) and \(b\) activate another protein \(c\) is insufficient because both
boolean functions \(a \land b\) and \(a \lor b\) are consistent with the
knowledge. This uncertainty may be expressed in a \emph{parametrised boolean
network} (PBN), a more general version of BN.

\paragraph{Attractors and control}

One crucial question about any system is: \enquote{How will the system look
after a long time? Will it settle and not change anymore?} The state the system
will eventually evolve into and never leave is called an \emph{attractor}. Such
a state does not have to be single; an attractor may also consist of multiple
states. In that case, the system switches among the states and visits each
infinitely often, but again, only those states in the attractor, no others.. In
the context of systems biology, an attractor is called a \emph{phenotype} as
well.

Finding a control strategy that drives the system to a desired attractor is
beneficial for many applications, for example, identifying key therapeutic
targets for controlling pathways of regulatory and signaling networks,
providing a basis for designing \emph{in vitro} cell reprogramming experiments,
and many more~\cite{control_psbn}. It is known that this problem is
\(\mathcal{NP}\)-hard in general for boolean networks, but effective
approximations exist ranging from linear to cubic time
complexities~\cite{control_akutsu}. Last year, T.~Parmer et~al. presented an
article~\cite{infl_max_BN} with a novel approximation approach based on a
well-studied problem of \emph{influence maximization} for spreading processes
in social networks.

It became a stimulus for this thesis. T.~Parmer later published
\cite{parmer_dynamical}~and~\cite{parmer_phd}, elaborating more on influence
and control in network-based models. However, to the best of our knowledge, no
attempt to generalize to parametrized boolean networks has been published yet.
Thus, in this work, we aim to adjust their idea to work with PBNs, benchmark on
a suitable set of networks, and design a simple, user-friendly GUI to analyze
models of the user's choice.

% TODO is this neccessary?
%\makeatletter\thesis@blocks@clear\makeatother
%\phantomsection %% Print the index and insert it into the
%\addcontentsline{toc}{chapter}{\indexname} %% table of contents.
%\printindex

\appendix
\chapter{An appendix}
Here you can insert the appendices of your thesis.

\end{document}
